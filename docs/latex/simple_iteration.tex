This example covers the basic usage of iterating over utf8 text codepoint by codepoint.


\begin{DoxyCode}

\textcolor{preprocessor}{#include <experimental/string\_view>}
\textcolor{preprocessor}{#include <u5e/utf8\_string\_view.hpp>}
\textcolor{preprocessor}{#include <stdio.h>}

\textcolor{keyword}{using} std::experimental::string\_view;

\textcolor{keywordtype}{int} main(\textcolor{keywordtype}{int} argc, \textcolor{keywordtype}{char} **argv) \{
  \textcolor{comment}{// for each argument}
  \textcolor{keywordflow}{for} (\textcolor{keywordtype}{int} i = 1; i < argc; i++) \{

    \textcolor{comment}{// get a string\_view}
    string\_view p(argv[i], strlen(argv[i]));

    \textcolor{comment}{// get a utf8\_string\_view::const\_iteator}
    \hyperlink{classu5e_1_1basic__encodedstring_a249da58e8bad9c91fab547516f90c60d}{u5e::utf8\_string\_view::const\_iterator} it = p.begin();

    \textcolor{comment}{// Iterate until the end}
    \textcolor{keywordflow}{while} (it != p.end()) \{

      \textcolor{comment}{// the value dereferenced is the codepoint, not octets even if}
      \textcolor{comment}{// the original text had "wide" chars.}
      printf(\textcolor{stringliteral}{" U+%06llx "}, (\textcolor{keywordtype}{long} \textcolor{keywordtype}{long} \textcolor{keywordtype}{unsigned} \textcolor{keywordtype}{int})*it++);
      
    \}
    
    printf(\textcolor{stringliteral}{"\(\backslash\)n"});
  \}
  \textcolor{keywordflow}{return} 0;
\}
\end{DoxyCode}
 