The challenge of truncating a string is that if you operate on the native level, you risk breaking a codepoint in half. And even when you take the codepoint boundaries in consideration, you still risk breaking a grapheme in half.

The \hyperlink{namespaceu5e}{u5e} library provides an easy way to truncate a text, starting from a target native size and finding the correct boundary.


\begin{DoxyCode}
\textcolor{preprocessor}{#include <algorithm>}
\textcolor{preprocessor}{#include <iostream>}
\textcolor{preprocessor}{#include <string>}
\textcolor{preprocessor}{#include <u5e/utf8\_string.hpp>}
\textcolor{preprocessor}{#include <u5e/utf8\_string\_grapheme\_iterator.hpp>}
\textcolor{keyword}{using} std::string;
\textcolor{keyword}{using} \hyperlink{namespaceu5e_a3814526c6ea7ca04223a8ec1b662d497}{u5e::utf8\_string};
\textcolor{keyword}{using} \hyperlink{namespaceu5e_a74a084fc31e6af27114007a21f6f89e7}{u5e::utf8\_string\_grapheme\_iterator};
\textcolor{keywordtype}{int} main(\textcolor{keywordtype}{int} argc, \textcolor{keywordtype}{char}** argv) \{
  \textcolor{comment}{// the original string}
  \textcolor{keywordtype}{string} str(\textcolor{stringliteral}{"Ola\(\backslash\)xCC\(\backslash\)x81!"});

  \textcolor{comment}{// we find the point on the native string where we want to truncate.}
  \textcolor{comment}{// Count 4 in octets leaves us in the middle of the second codepoint}
  \textcolor{comment}{// of a grapheme.}
  string::const\_iterator stri = str.cbegin();
  std::advance(stri, 4);

  \textcolor{comment}{// We need to promote the iterators to utf8 iterators.}
  utf8\_string::const\_iterator u8b(str.begin());
  utf8\_string::const\_iterator u8e(str.end());
  utf8\_string::const\_iterator u8i(stri);

  \textcolor{comment}{// In order to work with graphemes, we need to obtain grapheme}
  \textcolor{comment}{// iterators.}
  \hyperlink{namespaceu5e_a74a084fc31e6af27114007a21f6f89e7}{utf8\_string\_grapheme\_iterator} gb(u8b, u8e);
  \hyperlink{namespaceu5e_a74a084fc31e6af27114007a21f6f89e7}{utf8\_string\_grapheme\_iterator} gi(u8b, u8e, u8i);
  
  \textcolor{comment}{// now we produce the output string truncated at where we want.}
  \textcolor{comment}{// note that this assumes that the text was already in utf8. this is}
  \textcolor{comment}{// not a conversion operator, it just makes sure the append happens}
  \textcolor{comment}{// while respecting the codepoint and grapheme boundaries in the}
  \textcolor{comment}{// utf8 text.}
  \hyperlink{namespaceu5e_a3814526c6ea7ca04223a8ec1b662d497}{utf8\_string} output(\textcolor{stringliteral}{""});
  output.append(gb, gi);

  std::cout << output.native\_string << std::endl;
\}
\end{DoxyCode}
 